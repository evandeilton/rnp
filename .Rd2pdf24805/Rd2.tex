\documentclass[letterpaper]{book}
\usepackage[times,inconsolata,hyper]{Rd}
\usepackage{makeidx}
\usepackage[utf8]{inputenc} % @SET ENCODING@
% \usepackage{graphicx} % @USE GRAPHICX@
\makeindex{}
\begin{document}
\chapter*{}
\begin{center}
{\textbf{\huge Package `rnp'}}
\par\bigskip{\large \today}
\end{center}
\begin{description}
\raggedright{}
\inputencoding{utf8}
\item[Title]\AsIs{R NA PRATICA - Pacote de recursos extras}
\item[Version]\AsIs{0.0.1}
\item[Description]\AsIs{rnp é a sigla para <R NA PRÁTICA>. Este pacote é um combo extra de funções e recursos para todos os alunos do rnp. Este pacote está em atualização contínua e novos resursos serão adicionados.}
\item[License]\AsIs{GPL-3}
\item[Encoding]\AsIs{UTF-8}
\item[Suggests]\AsIs{knitr,
rmarkdown,
stringr,
dplyr}
\item[VignetteBuilder]\AsIs{knitr}
\item[LazyData]\AsIs{true}
\item[RoxygenNote]\AsIs{6.1.1}
\end{description}
\Rdcontents{\R{} topics documented:}
